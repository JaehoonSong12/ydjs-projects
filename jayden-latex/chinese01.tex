% chinese_studyguide_pdflatex.tex
% PDFLaTeX (default engine) version of the expanded study guide
\documentclass[12pt]{article}
\usepackage[utf8]{inputenc}
\usepackage[T1]{fontenc}
\usepackage{CJKutf8}        % <-- CJK support for pdfLaTeX
\usepackage[a4paper,margin=0.5in]{geometry}
\usepackage{booktabs}
\usepackage{longtable}
\usepackage{multicol}
\usepackage{enumitem}
\usepackage{hyperref}
\usepackage{parskip}
\hypersetup{colorlinks=true,linkcolor=blue}

\title{Chinese Quiz — Expanded Study Guide (pdfLaTeX / CJKutf8)}
\author{}
\date{}

\begin{document}
\maketitle

% Begin CJK environment for Chinese text (UTF8) and choose a built-in font family.
% Common family choices: gbsn (GBSong, Simplified), gkai (GBKai), bsmi (Big5/Traditional).
% If gbsn is missing try gkai or bsmi, or install texlive-lang-chinese.
\begin{CJK}{UTF8}{gbsn}

\section*{1. Quick Reference: Characters with Pinyin (from your quiz)}
\begin{longtable}{lll}
\toprule
\# & Character & Pinyin / Quick meaning \\
\midrule
1 & 旦 & dān — dawn, daybreak \\
2 & 杲 & gǎo — bright (rare, literary) \\
3 & 从 & cóng — from, follow \\
4 & 并 & bìng — and, moreover; together \\
5 & 北 & běi — north (e.g., 北京 Běijīng) \\
6 & 夭 & yāo — young; to die young (classical) \\
7 & 儿 & ér — child; also the -er suffix in many words \\
8 & 晚 & wǎn — late, evening \\
9 & 欠 & qiàn — owe; to lack; (radical often shows mouth open/yawning) \\
10 & 暮 & mù — dusk; evening \\
11 & 众 & zhòng — crowd; many \\
12 & 杳 & yǎo — deep/unknown (literary) \\
13 & 化 & huà — change; transform (also suffix -ize) \\
14 & 夫 & fū — man; husband (in compounds like 丈夫 zhàngfu) \\
15 & 尸 & shī — corpse (radical in some characters) \\
16 & 昏 & hūn — dark; dizzy; to faint; dusk \\
17 & 子 & zǐ — child; suffix; noun marker \\
18 & 老 & lǎo — old; used in titles (老师 lǎoshī) \\
19 & 旭 & xù — rising sun; dawn \\
20 & 了 & le — (aspect particle) / completed action \\
21 & 立 & lì — to stand; establish \\
22 & 交 & jiāo — to exchange; to cross; to hand over \\
23 & 夫 & fū — male adult / husband (duplicate entry) \\
24 & 老 & lǎo — old (duplicate entry) \\
25 & 卩 & jié (radical) — seal; kneel (radical form) \\
26 & 文 & wén — writing; culture; language \\
27 & 早 & zǎo — early; morning \\
28 & 晚 & wǎn — evening / late (duplicate) \\
29 & 大 & dà — big; large \\
30 & 北 & běi — north (duplicate) \\
\bottomrule
\end{longtable}

\section*{2. Common Compounds \& Example Words (learn characters in context)}
\begin{itemize}
  \item 北 (běi): 北京 (Běijīng), 北方 (běifāng — north)
  \item 早 (zǎo): 早上 (zǎoshang — morning), 早点 (zǎodiǎn — breakfast; lit. early-spot)
  \item 晚 (wǎn): 晚上 (wǎnshang — evening), 晚安 (wǎn'ān — good night)
  \item 老 (lǎo): 老师 (lǎoshī — teacher), 老人 (lǎorén — elder)
  \item 子 (zǐ): 孩子 (háizi — child), 子女 (zǐnǚ — children)
  \item 化 (huà): 变化 (biànhuà — change), 化学 (huàxué — chemistry)
  \item 立 (lì): 立刻 (lìkè — immediately), 建立 (jiànlì — establish)
  \item 从 (cóng): 从来 (cónglái — always/as long as), 从前 (cóngqián — once upon a time)
  \item 文 (wén): 文化 (wénhuà — culture), 文学 (wénxué — literature)
  \item 并 (bìng): 并且 (bìngqiě — moreover), 并非 (bìngfēi — actually not)
  \item 旦 / 旭: 旦 (dān) and 旭 (xù) both relate to sun/dawn — useful pair for morning/dawn words
\end{itemize}

\section*{3. Radicals \& Why they help}
Many characters share parts (radicals). Recognizing radicals speeds up meaning guessing and writing.

\begin{itemize}
  \item \textbf{日} (sun/day): appears in 暮、旦、旭、昏 — linked to time of day or sunlight.
  \item \textbf{儿} (child/suffix) — appears in 儿 and in many derived forms.
  \item \textbf{欠} (yawn/owe) — hints at mouth/opening / lacking / yawning; appears in 欠 itself and related characters.
  \item \textbf{卩} (kneel/seal) — simple radical in characters like 印.
  \item \textbf{文} (pattern/writing) — appears in cultural/word-related characters.
\end{itemize}

\section*{4. Short Mnemonics (one line each) — say them out loud while writing}
\begin{itemize}[leftmargin=*]
  \item 旦 (dān): ``sun on a plate'' — dawn (sun rising in a circle).  
  \item 杲 (gǎo): ``two trees + sun = very bright'' (rare word for bright).
  \item 从 (cóng): two people following each other → follow/from.
  \item 并 (bìng): two items side-by-side = together / and.
  \item 北 (běi): looks like two people back-to-back → north (direction).
  \item 夭 (yāo): a bent person — short life / young (classical).
  \item 儿 (ér): child shape — son/child.
  \item 晚 (wǎn): sun + evening sign = evening/late.
  \item 欠 (qiàn): open mouth yawning = owe / lack.
  \item 暮 (mù): sunset + grass roof = dusk.
  \item 众 (zhòng): three people → many.
  \item 杳 (yǎo): deep/hidden — think ``deep forest''.
  \item 化 (huà): person changing shape = change / transform.
  \item 夫 (fū): man with broad hat → man / husband.
  \item 尸 (shī): corpse shape → dead / body radical.
  \item 昏 (hūn): sun + mix = dusk / faint.
  \item 子 (zǐ): small child head — child.
  \item 老 (lǎo): old man with long hair/beard (visual).
  \item 旭 (xù): rising sun → dawn.
  \item 了 (le): a small hook — particle showing change/completion.
  \item 立 (lì): person standing on ground → to stand.
  \item 交 (jiāo): crossing lines → to cross / exchange.
  \item 文 (wén): a pattern on cloth → culture / writing.
  \item 大 (dà): person with arms wide = big.
\end{itemize}

\section*{5. Pronunciation Tips (Pinyin \& Tones) — practice naturally}
\begin{enumerate}
  \item \textbf{Say out loud.} Read each character + pinyin + tone: e.g., \textbf{北京 Běijīng} — say the tones clearly.
  \item \textbf{Tone practice}: Repeat simple words across four tones: \textit{mā, má, mǎ, mà} — practice to feel high, rising, dipping, falling tones.
  \item \textbf{Tone pairs}: practice two-syllable pairs like: \textbf{běi + jīng}, \textbf{zǎo + shang}, \textbf{wǎn + shang}. Record yourself and compare.
  \item \textbf{Tone sandhi}: note that \textbf{不} (bù) may change to \textbf{bú} before a fourth tone; and consecutive third tones change in natural speech.
  \item \textbf{Pronounce finals clearly}: exaggerate finals like \textit{ao, an, eng, iu} slowly then speed up.
\end{enumerate}

\section*{6. Natural Memorization Plan (use every day) — simple SRS + active recall}
\textbf{Daily 15--25 minutes} split into short blocks:
\begin{enumerate}
  \item \textbf{Warm-up (3--5 min):} Read the list out loud. Say pinyin + meaning.
  \item \textbf{Active recall (7--10 min):} Cover the pinyin/meaning and \emph{write} the character from memory (or say it). Then check.
  \item \textbf{Production (5--10 min):} Use each character in a short word or sentence. Speak the sentence aloud.
  \item \textbf{Quick review (2 min):} Mark which ones were hard (red), okay (yellow), easy (green).
\end{enumerate}

\textbf{Spaced schedule (SRS) for each card:} review after \textbf{1 day, 3 days, 7 days, 14 days, 1 month, 3 months}.  
Use an app (Anki) or a simple notebook with dates.

\section*{7. Active practice exercises (do them aloud and in writing)}
\begin{enumerate}
  \item \textbf{Matching (10 min):} Write characters on left column and pinyin+meaning on right shuffled — match without looking.
  \item \textbf{Write from pinyin:} I say ``běi'' → write 北 and make the example word 北京.
  \item \textbf{Mini sentences:} Make one sentence for each of these: 北, 早, 晚, 老, 子, 化. (E.g., 我早上七点起床。)
  \item \textbf{Flashcard test:} 30 sec per card — say pinyin, meaning, and use it in a short phrase.
  \item \textbf{Tone-only drill:} For a list of 10 syllables, only speak the tones (e.g., 3rd, 1st, 4th...), then speak full pinyin.
\end{enumerate}

\section*{8. Example short sentences (useful for items 31--36)}
\begin{itemize}
  \item 你是哪国人?\quad 我是美国人。 (What nationality are you? I am American.)
  \item 你上哪个学校?\quad 我上古内数理与技术高中。 (Which school? I go to \dots)
  \item 你家有几口人?\quad 我家有九口人。 (How many people? I have nine people in my family.)
  \item 你有姐姐吗?\quad 有,我有一个姐姐。 / 没有,我没有姐姐。 (Do you have an older sister?)
  \item 她是谁?\quad 她是我的奶奶。 (Who is she? She is my grandmother.)
  \item 他是谁?\quad 他是我的外公。 (Who is he? He is my grandfather.)
\end{itemize}

\section*{9. Flashcard template (print or write by hand)}
\begin{multicols}{2}
\begin{enumerate}[leftmargin=*]
  \item \textbf{Front:} Character (e.g., 北) \\
        \textbf{Back:} 1) Pinyin: běi \quad 2) Meaning: north \quad 3) Example: 北京 Běijīng
  \item \textbf{Front:} Pinyin + tone (e.g., wǎn) \\
        \textbf{Back:} Character: 晚 \quad 2) Example: 晚上 wǎnshang
  \item \textbf{Front:} English (e.g., morning) \\
        \textbf{Back:} 早 zǎo, 早上 zǎoshang
\end{enumerate}
\end{multicols}

\section*{10. How to review so it ``sticks'' (few easy rules)}
\begin{itemize}
  \item \textbf{Write by hand.} Motor memory helps characters stick.
  \item \textbf{Speak and listen.} Record yourself and compare; say words in phrases.
  \item \textbf{Use the character in a short sentence every day.}
  \item \textbf{Mix:} Don’t study only what you know — spend more time on the hard ones.
  \item \textbf{Test yourself under time pressure.} 60 seconds to recall 10 cards builds confidence.
\end{itemize}

\section*{11. Small cheatsheet: Tone symbols (for reading pinyin)}
\begin{tabular}{ll}
ā (1st) & high flat \\
á (2nd) & rising (like a question) \\
ǎ (3rd) & low dip (fall-rise) — often becomes low in fast speech \\
à (4th) & sharp falling \\
\end{tabular}

\section*{12. Quick trouble-shoot: similar characters}
\begin{itemize}
  \item 早 (zǎo) vs 暮 (mù): 早 = morning, 暮 = dusk/evening. Note the difference: 早 has a sun and a small top; 暮 includes grass + sun.
  \item 老 (lǎo) vs 考 (kǎo): note hair/beard strokes for 老.
  \item 子 (zǐ) vs 孑 (jié): small stroke differences — slow careful writing helps.
\end{itemize}

\section*{13. Printable one-week micro-plan (15 min/day)}
\begin{enumerate}
  \item Day 1: Read all 36 aloud; write 10 chosen hard ones.
  \item Day 2: Flashcards: active recall + 5 example words.
  \item Day 3: Write 20 times each of 5 hardest characters; speak sentences.
  \item Day 4: Mixed review: matching quiz + tone practice 10 min.
  \item Day 5: Use characters in 6 short sentences; record yourself.
  \item Day 6: Timed quiz (2 mins for 12 cards) + SRS marking.
  \item Day 7: Free practice: read a short paragraph or make a mini-dialog using learned words.
\end{enumerate}

\section*{14. Final tips (short)}
\begin{itemize}
  \item Learn a few new chars every day, but review old ones. \\
  \item Always use pinyin+tone when you first learn the character. \\
  \item Make silly pictures for mnemonics — funny images stick best. \\
  \item Writing correctly (stroke order) makes memorization faster — if you want, I can add a stroke-order list for the top 12 characters next.
\end{itemize}

\bigskip
\noindent\textbf{If you want:} I can generate (right now) \\
\quad • stroke-order lists for the top 12 characters, \\
\quad • 30 printable flashcards in pdfLaTeX format, or \\
\quad • a short timed quiz (10 questions) you can print and use. \\
Tell me which one and I’ll append it into this same .tex file.

\end{CJK}

\end{document}
