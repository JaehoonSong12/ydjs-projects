% Compile with pdfLaTeX (default engine)
\documentclass[12pt]{article}
\usepackage[utf8]{inputenc}
\usepackage[T1]{fontenc}
\usepackage{CJKutf8}        % <-- CJK support for pdfLaTeX
\usepackage[a4paper,margin=1in]{geometry}
\usepackage{booktabs}
\usepackage{longtable}
\usepackage{enumitem}
\usepackage{hyperref}
\hypersetup{colorlinks=true,linkcolor=blue}

\title{Chinese Quiz — Study Guide (pdfLaTeX / CJKutf8)}
\author{}
\date{}

\begin{document}
\maketitle

% Begin CJK environment for Chinese text (UTF8) and choose a built-in font family.
% Common family choices: gbsn (GBSong, Simplified), gkai (GBKai), bsmi (Big5/Traditional).
% If gbsn is missing try gkai, bsmi, or install texlive-lang-chinese.
\begin{CJK}{UTF8}{gbsn}

\section*{1. Quick reference: Characters with Pinyin}
\begin{longtable}{llp{8cm}}
\toprule
\# & Character & Pinyin / Quick meaning \\
\midrule
1 & 旦 & dān — dawn, daybreak \\
2 & 杲 & gǎo — bright (rare, literary) \\
3 & 从 & cóng — from, follow \\
4 & 并 & bìng — and; together \\
5 & 北 & běi — north \\
6 & 夭 & yāo — young / die young (classical) \\
7 & 儿 & ér — child; -er suffix \\
8 & 晚 & wǎn — late; evening \\
9 & 欠 & qiàn — owe; to lack \\
10 & 暮 & mù — dusk; evening \\
11 & 众 & zhòng — crowd; many \\
12 & 杳 & yǎo — hidden / deep (literary) \\
13 & 化 & huà — change; transform \\
14 & 夫 & fū — man; husband \\
15 & 尸 & shī — corpse (radical) \\
16 & 昏 & hūn — dark; faint; dusk \\
17 & 子 & zǐ — child; suffix \\
18 & 老 & lǎo — old \\
19 & 旭 & xù — rising sun; dawn \\
20 & 了 & le — aspect particle / completed action \\
\bottomrule
\end{longtable}

\section*{2. Useful compounds \& example words}
北 (běi): 北京 Běijīng, 北方 běifāng \\
早 (zǎo): 早上 zǎoshang, 早点 zǎodiǎn \\
晚 (wǎn): 晚上 wǎnshang, 晚安 wǎn'ān \\
老 (lǎo): 老师 lǎoshī, 老人 lǎorén \\
子 (zǐ): 孩子 háizi, 子女 zǐnǚ \\
化 (huà): 变化 biànhuà, 化学 huàxué

\section*{3. Sample sentences (31--36)}
你是哪国人? 我是美国人。 \\
你上哪个学校? 我上古内数理与技术高中。 \\
你家有几口人? 我家有九口人。 \\
你有姐姐吗? 有,我有一个姐姐。 \\
她是谁? 她是我的奶奶。 \\
他是谁? 他是我的外公。

\end{CJK}

%   \item If you see empty squares where Chinese should be, install \texttt{texlive-lang-chinese} (TeX Live) or equivalent CJK fonts for your TeX distribution.
%   \item If the font family \texttt{gbsn} is not available, change the `\{gbsn\}` in `\begin{CJK}{UTF8}{gbsn}` to `gkai` or `bsmi`, or install the language package noted above.
%   \item For nicer system fonts and simpler Unicode handling, XeLaTeX is recommended — but this document is intentionally compatible with pdfLaTeX per your request.


\end{document}
