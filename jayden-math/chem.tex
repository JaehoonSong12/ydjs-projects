\documentclass{article}
\usepackage{tikz}
\usepackage{chemfig}
\usepackage{xcolor}
\usepackage{enumitem}
\usepackage{amsthm}
\usepackage{amsmath}
\usepackage{amssymb}
\usepackage{environ}
\usepackage{pgfplots}
\pgfplotsset{compat=1.18}

\usepackage{mathtools}
\usepackage{tabularx}
\usepackage[
    top=    0.50in,
    bottom= 0.50in,
    left=   0.50in,
    right=  0.50in,
]{geometry}

\usepackage{hyperref}

% no indent
\setlength\parindent{0pt}

% Theorem environments
\theoremstyle{definition}
\newtheorem{definition}{Definition}
\newtheorem{theorem}{Theorem}
\newtheorem{lemma}{Lemma}
\newtheorem{proposition}{Proposition}
\newtheorem{example}{Example}
\newtheorem{problem}{Problem}
\newtheorem{corollary}{Corollary}

\NewEnviron{answer}{%
  \par\noindent\vspace{10pt}%
  \textcolor{red}{\textbf{Answer}:}
  \par\noindent\textcolor{red}{\BODY}%
  \par\vspace{-0.5\baselineskip}\raggedleft\textcolor{red}{$\blacksquare$}%
  \par\vspace{10pt}
}









\begin{document}











\section*{1. Characteristics of Ionic, Covalent, and Metallic Compounds}

\begin{table}[h]
\centering
\begin{tabular}{|l|p{4cm}|p{4cm}|p{4cm}|}
\hline
\textbf{Property} & \textbf{Ionic} & \textbf{Covalent (Molecular)} & \textbf{Metallic} \\
\hline
\textbf{Formation} & Metal + nonmetal & Two or more nonmetals & Metal atoms only \\
\hline
\textbf{Electron behavior} & Transferred from metal to nonmetal & Shared between atoms & ``Sea of mobile electrons'' \\
\hline
\textbf{Melting/Boiling points} & High & Low & Variable (generally high) \\
\hline
\textbf{Electrical conductivity} & When molten or dissolved (aq) & Do not conduct & Conduct heat and electricity \\
\hline
\textbf{Physical properties} & Hard, brittle crystalline solids & Exist as individual molecules & Malleable, ductile, shiny \\
\hline
\end{tabular}
\end{table}

\section*{2. Identifying Bond Type Using Electronegativity Difference}

Let $\Delta EN = |EN_1 - EN_2|$.

\begin{table}[h]
\centering
\begin{tabular}{|c|l|}
\hline
\textbf{$\Delta EN$ Range} & \textbf{Bond Type} \\
\hline
$0$ to $0.4$ & Nonpolar covalent \\
\hline
$0.5$ to $1.7$ & Polar covalent \\
\hline
$> 1.7$ & Ionic \\
\hline
\end{tabular}
\end{table}

\textbf{General rule:} metal + nonmetal $\Rightarrow$ ionic;  
nonmetal + nonmetal $\Rightarrow$ covalent.

\section*{3. Oxidation States of Common Groups}

\begin{table}[h]
\centering
\begin{tabular}{|l|c|}
\hline
\textbf{Group} & \textbf{Oxidation State} \\
\hline
Group 1 & +1 \\
\hline
Group 2 & +2 \\
\hline
Group 13 & +3 \\
\hline
Group 15 & --3 \\
\hline
Group 16 & --2 \\
\hline
Group 17 & --1 \\
\hline
\end{tabular}
\end{table}

\textbf{Special cases:}
\begin{table}[h]
\centering
\begin{tabular}{|l|l|}
\hline
\textbf{Element} & \textbf{Oxidation State} \\
\hline
Hydrogen & +1 (usually), --1 in metal hydrides \\
\hline
Oxygen & --2, except in peroxides (--1) \\
\hline
Transition metals & Multiple oxidation states \\
\hline
\end{tabular}
\end{table}

\section*{4. Covalent vs. Ionic Bonding Summary}

\subsection*{Ionic Bonding}
\begin{itemize}
    \item Electrons transferred.
    \item Attraction between cations and anions.
\end{itemize}

\subsection*{Covalent Bonding}
\begin{itemize}
    \item Electrons shared.
    \item Bond order: single, double, triple.
\end{itemize}

\section*{5. Lewis Dot Structures (LDS)}

Steps to draw:
\begin{enumerate}
    \item Count all valence electrons.
    \item Draw a skeleton (central atom is least electronegative, never H).
    \item Add single bonds.
    \item Complete octets on outer atoms first.
    \item Place remaining electrons on central atom.
    \item Form double or triple bonds if needed to satisfy octet.
\end{enumerate}

\textbf{Octet rule exceptions:}
\begin{itemize}
    \item Incomplete octet: B, Be.
    \item Expanded octet: P, S, Cl, Br, Xe (period 3+).
    \item Odd-electron molecules (e.g., NO).
\end{itemize}

\section*{6. VSEPR Theory: Molecular Shapes}

VSEPR formula: AX$_m$E$_n$ (A = central atom, X = bonded atoms, E = lone pairs).

\begin{table}[h]
\centering
\begin{tabular}{|c|l|c|}
\hline
\textbf{VSEPR Formula} & \textbf{Molecular Shape} & \textbf{Bond Angle} \\
\hline
AX$_2$ & Linear & 180$^\circ$ \\
\hline
AX$_3$ & Trigonal planar & 120$^\circ$ \\
\hline
AX$_2$E & Bent & 119$^\circ$ \\
\hline
AX$_4$ & Tetrahedral & 109.5$^\circ$ \\
\hline
AX$_3$E & Pyramidal & 107$^\circ$ \\
\hline
AX$_2$E$_2$ & Bent & 105$^\circ$ \\
\hline
AX$_5$ & Trigonal bipyramidal & -- \\
\hline
AX$_6$ & Octahedral & -- \\
\hline
\end{tabular}
\end{table}








\begin{table}[h!]
  \centering
  \small
  \begin{tabularx}{\textwidth}{|p{1.8cm}|p{2.2cm}|p{2.5cm}|X|}
  \hline
  \textbf{Chemical Formula} & \textbf{Central Atom} & \textbf{AX$_m$E$_n$} & \textbf{Why the Formula Order Is Written This Way} \\
  \hline
  H$_2$O & O & AX$_2$E$_2$ & H written first because it is less electronegative than O. \\
  \hline
  NH$_3$ & N & AX$_3$E$_1$ & H written after N due to naming convention; N is more common central atom. \\
  \hline
  CH$_4$ & C & AX$_4$E$_0$ & C is always central in organic compounds; H never central. \\
  \hline
  CO$_2$ & C & AX$_2$E$_0$ & By convention, C is written before O (less electronegative). \\
  \hline
  SO$_2$ & S & AX$_2$E$_1$ & S is less electronegative, so written before O. \\
  \hline
  HCN & C & AX$_2$E$_0$ & Written H$-$C$\equiv$N; C is central but appears in the middle of the formula. \\
  \hline
  CH$_3$OH & C (O terminal) & AX$_4$E$_0$ (C); AX$_2$E$_2$ (O) & Formula follows organic grouping (CH$_3$)(OH). \\
  \hline
  HNO$_3$ & N & AX$_3$E$_0$ (approx.) & Acid formulas begin with H even though N is central. \\
  \hline
  H$_2$SO$_4$ & S & AX$_4$E$_0$ (expanded) & Acids list H first; central atom (S) appears later. \\
  \hline
  OCl$_2$ & O & AX$_2$E$_2$ & O is less common as central; formula still writes O first due to electronegativity pattern. \\
  \hline
  ClO$_2$ & Cl & AX$_2$E$_2$ & Written ClO$_2$ because Cl is central despite being more electronegative than O. \\
  \hline
  PCl$_3$ & P & AX$_3$E$_1$ & P is central atom but formula follows naming order. \\
  \hline
  \end{tabularx}
  \caption{Examples Showing Formula Order vs.\ VSEPR Central Atom}
\end{table}
  

\newpage

\section*{7. Naming Ionic, Covalent, Polyatomic, and Acid Compounds}

\begin{table}[h]
\centering
\begin{tabular}{|l|p{5cm}|p{4cm}|}
\hline
\textbf{Compound Type} & \textbf{Naming Rule} & \textbf{Example} \\
\hline
\textbf{Ionic} & Metal name + nonmetal root + ``ide'' & NaCl = sodium chloride \\
\hline
\textbf{Ionic (transition metal)} & Metal name (Roman numeral) + nonmetal root + ``ide'' & FeCl$_3$ = iron(III) chloride \\
\hline
\textbf{Polyatomic} & Keep ion name intact & NaNO$_3$ = sodium nitrate \\
\hline
\textbf{Covalent} & Use prefixes: mono-, di-, tri-, tetra-, penta-, hexa- & CO$_2$ = carbon dioxide \\
\hline
\textbf{Binary acid} & ``hydro'' + root + ``ic acid'' & HCl(aq) = hydrochloric acid \\
\hline
\textbf{Oxyacid (\textit{ate})} & stem\textit{ic} acid & HNO$_3$ = nitric acid \\
\hline
\textbf{Oxyacid (\textit{ite})} & stem\textit{ous} acid & HNO$_2$ = nitrous acid \\
\hline
\end{tabular}
\end{table}

\textbf{Note:} Polyatomic ions: \textit{ate} ending = more oxygen; \textit{ite} ending = less oxygen.

\section*{8. Writing Formulas from Names}

\begin{itemize}
    \item Identify cation and anion.
    \item Balance charges to make the compound neutral.
    \item Place polyatomic ions in parentheses if more than one is needed:
          Ca(OH)$_2$.
\end{itemize}

\section*{9. Writing Names from Formulas}

\begin{itemize}
    \item Ionic: metal name + nonmetal ending in \textit{ide}.
    \item Covalent: use prefixes.
    \item Polyatomic: keep ion name.
    \item Acids: follow acid naming rules.
\end{itemize}





\newpage

\section*{Bonding and Naming Practice (Lewis Structures)}

%%%%%%%%%%%%%%%%%%%%%%%%%%%%%%%%%%%%%%%%%%%%%%%%%%%%%%%%%%%%
% Problem 1
%%%%%%%%%%%%%%%%%%%%%%%%%%%%%%%%%%%%%%%%%%%%%%%%%%%%%%%%%%%%
\begin{problem}
Ionic bonds are normally formed when: \\
A. electrons are shared between a metal and a nonmetal \\
B. electrons are shared between two nonmetals \\
C. electrons are transferred from a metal to a nonmetal \\
D. electrons are transferred from a nonmetal to a metal
\end{problem}

\begin{answer}
Correct: C \\
A metal transfers electrons to a nonmetal in ionic bonding.
\end{answer}

%%%%%%%%%%%%%%%%%%%%%%%%%%%%%%%%%%%%%%%%%%%%%%%%%%%%%%%%%%%%
% Problem 2
%%%%%%%%%%%%%%%%%%%%%%%%%%%%%%%%%%%%%%%%%%%%%%%%%%%%%%%%%%%%
\begin{problem}
Covalent bonds are normally formed when: \\
A. electrons are shared between a metal and a nonmetal \\
B. electrons are shared between two nonmetals \\
C. electrons are transferred from a metal to a nonmetal \\
D. electrons are transferred from a nonmetal to a metal
\end{problem}

\begin{answer}
Correct: B \\
Two nonmetals share electrons.
\end{answer}

%%%%%%%%%%%%%%%%%%%%%%%%%%%%%%%%%%%%%%%%%%%%%%%%%%%%%%%%%%%%
% Problem 3
%%%%%%%%%%%%%%%%%%%%%%%%%%%%%%%%%%%%%%%%%%%%%%%%%%%%%%%%%%%%
\begin{problem}
Which compound is ionic? \\
A. CO2 \quad B. ZnCl2 \quad C. SF2 \quad D. SeBr2
\end{problem}

\begin{answer}
Correct: B \\
Zinc is a metal; chlorine is a nonmetal.
\end{answer}

%%%%%%%%%%%%%%%%%%%%%%%%%%%%%%%%%%%%%%%%%%%%%%%%%%%%%%%%%%%%
% Problem 4
%%%%%%%%%%%%%%%%%%%%%%%%%%%%%%%%%%%%%%%%%%%%%%%%%%%%%%%%%%%%
\begin{problem}
Which compound is covalent? \\
A. PF3 \quad B. CNiBr3 \quad C. GaCl3 \quad D. CrO3
\end{problem}

\begin{answer}
Correct: A \\

Below is a Lewis structure for PF3 (trigonal pyramidal):

\begin{center}
\chemfig{P(-[:90]\charge{45=\|,225=\|,135=\|,315=\|}{F})(-[:210]\charge{45=\|,225=\|,135=\|,315=\|}{F})(-[:330]\charge{45=\|,225=\|,135=\|,315=\|}{F})\charge{90=\|,270=\|}{}}
\end{center}

\end{answer}

%%%%%%%%%%%%%%%%%%%%%%%%%%%%%%%%%%%%%%%%%%%%%%%%%%%%%%%%%%%%
% Problem 5
%%%%%%%%%%%%%%%%%%%%%%%%%%%%%%%%%%%%%%%%%%%%%%%%%%%%%%%%%%%%
\begin{problem}
Which compound requires a Roman numeral? \\
A. SF6 \quad B. AlBr3 \quad C. ZnO \quad D. PdCl2
\end{problem}

\begin{answer}
Correct: D \\
Palladium is a transition metal with variable oxidation states.
\end{answer}


\newpage


%%%%%%%%%%%%%%%%%%%%%%%%%%%%%%%%%%%%%%%%%%%%%%%%%%%%%%%%%%%%
% Problem 6
%%%%%%%%%%%%%%%%%%%%%%%%%%%%%%%%%%%%%%%%%%%%%%%%%%%%%%%%%%%%
\begin{problem}
The correct formula for strontium phosphide is: \\
A. Sr3P \quad B. SrPO4 \quad C. Sr3P2 \quad D. Sr3(PO4)2
\end{problem}

\begin{answer}
Correct: C \\
Sr is 2+ and P is 3-, producing Sr3P2.
\end{answer}

%%%%%%%%%%%%%%%%%%%%%%%%%%%%%%%%%%%%%%%%%%%%%%%%%%%%%%%%%%%%
% Problem 7
%%%%%%%%%%%%%%%%%%%%%%%%%%%%%%%%%%%%%%%%%%%%%%%%%%%%%%%%%%%%
\begin{problem}
The correct formula for aluminum sulfide is: \\
A. Al2S3 \quad B. AlSO4 \quad C. Al5S2 \quad D. Al2(SO4)3
\end{problem}

\begin{answer}
Correct: A \\
Al is 3+ and S is 2-.
\end{answer}

%%%%%%%%%%%%%%%%%%%%%%%%%%%%%%%%%%%%%%%%%%%%%%%%%%%%%%%%%%%%
% Problem 8
%%%%%%%%%%%%%%%%%%%%%%%%%%%%%%%%%%%%%%%%%%%%%%%%%%%%%%%%%%%%
\begin{problem}
The correct formula for calcium hydroxide is: \\
A. CaO \quad B. CaOH2 \quad C. CaH2 \quad D. Ca(OH)2
\end{problem}

\begin{answer}
Correct: D \\
Calcium (2+) requires two hydroxide ions.
\end{answer}

%%%%%%%%%%%%%%%%%%%%%%%%%%%%%%%%%%%%%%%%%%%%%%%%%%%%%%%%%%%%
% Problem 9
%%%%%%%%%%%%%%%%%%%%%%%%%%%%%%%%%%%%%%%%%%%%%%%%%%%%%%%%%%%%
\begin{problem}
The correct name for Na3N is: \\
A. sodium nitride \\
B. trisodium mononitride \\
C. sodium(III) nitride \\
D. sodium nitrate
\end{problem}

\begin{answer}
Correct: A \\
Standard ionic naming: sodium nitride.
\end{answer}

%%%%%%%%%%%%%%%%%%%%%%%%%%%%%%%%%%%%%%%%%%%%%%%%%%%%%%%%%%%%
% Problem 10
%%%%%%%%%%%%%%%%%%%%%%%%%%%%%%%%%%%%%%%%%%%%%%%%%%%%%%%%%%%%
\begin{problem}
The correct name for CaCl2 is: \\
A. calcium(II) chloride \\
B. calcium chloride \\
C. calcium dichloride \\
D. calcium chlorate
\end{problem}

\begin{answer}
Correct: B \\
Ca forms only Ca2+.
\end{answer}

%%%%%%%%%%%%%%%%%%%%%%%%%%%%%%%%%%%%%%%%%%%%%%%%%%%%%%%%%%%%
% Problem 11
%%%%%%%%%%%%%%%%%%%%%%%%%%%%%%%%%%%%%%%%%%%%%%%%%%%%%%%%%%%%
\begin{problem}
The correct formula for sodium carbonate is: \\
A. NaC \quad B. Na2CO3 \quad C. NaCO3 \quad D. Na3CO3
\end{problem}

\begin{answer}
Correct: B \\
CO3 has a 2- charge, needing two Na+ ions.
\end{answer}




\newpage

%%%%%%%%%%%%%%%%%%%%%%%%%%%%%%%%%%%%%%%%%%%%%%%%%%%%%%%%%%%%
% Problem 12
%%%%%%%%%%%%%%%%%%%%%%%%%%%%%%%%%%%%%%%%%%%%%%%%%%%%%%%%%%%%
\begin{problem}
The correct name for Mg(NO3)2 is: \\
A. magnesium nitride \\
B. magnesium nitrate \\
C. magnesium dinitrate \\
D. magnesium(II) nitrate
\end{problem}

\begin{answer}
Correct: B \\
Nitrate is NO3-. Magnesium is always 2+.
\end{answer}

%%%%%%%%%%%%%%%%%%%%%%%%%%%%%%%%%%%%%%%%%%%%%%%%%%%%%%%%%%%%
% Problem 13
%%%%%%%%%%%%%%%%%%%%%%%%%%%%%%%%%%%%%%%%%%%%%%%%%%%%%%%%%%%%
\begin{problem}
The correct formula for dinitrogen trioxide is: \\
A. N2O \quad B. N2O4 \quad C. N2O3 \quad D. N3O2
\end{problem}

\begin{answer}
Correct: C \\

Lewis structure for N$_2$O$_3$ (one simplified resonance form):

\begin{center}
\chemfig{\charge{45=\|,225=\|,135=\|,315=\|}{O}=N(-[:90]\charge{45=\|,225=\|,135=\|,315=\|}{O})-N=\charge{45=\|,225=\|,135=\|,315=\|}{O}}
\end{center}

\end{answer}






%%%%%%%%%%%%%%%%%%%%%%%%%%%%%%%%%%%%%%%%%%%%%%%%%%%%%%%%%%%%
% PROBLEM 14
%%%%%%%%%%%%%%%%%%%%%%%%%%%%%%%%%%%%%%%%%%%%%%%%%%%%%%%%%%%%
\begin{problem}
The correct name for SF4 is: \\
A. sulfur(IV) fluoride \\
B. sulfur fluoride(IV) \\
C. sulfur trifluoride \\
D. sulfur tetrafluoride
\end{problem}

\begin{answer}
Correct: D \\
SF4 uses covalent naming. Four fluorine atoms give the prefix "tetra."
\end{answer}



%%%%%%%%%%%%%%%%%%%%%%%%%%%%%%%%%%%%%%%%%%%%%%%%%%%%%%%%%%%%
% PROBLEM 15
%%%%%%%%%%%%%%%%%%%%%%%%%%%%%%%%%%%%%%%%%%%%%%%%%%%%%%%%%%%%
\begin{problem}
Which of the following choices has classified the bonds correctly?
\end{problem}

\begin{answer}
Correct: D \\

H--O is covalent (nonmetal + nonmetal). \\
Ca--N is ionic (metal + nonmetal).
\end{answer}



%%%%%%%%%%%%%%%%%%%%%%%%%%%%%%%%%%%%%%%%%%%%%%%%%%%%%%%%%%%%
% PROBLEM 16
%%%%%%%%%%%%%%%%%%%%%%%%%%%%%%%%%%%%%%%%%%%%%%%%%%%%%%%%%%%%
\begin{problem}
As a bond between a hydrogen atom and a sulfur atom is formed, electrons are: \\
A. Shared to form an ionic bond \\
B. Shared to form a covalent bond \\
C. Transferred to form an ionic bond \\
D. Transferred to form a covalent bond
\end{problem}

\begin{answer}
Correct: B \\
Hydrogen and sulfur are both nonmetals, forming covalent bonds by electron sharing.
\end{answer}




\newpage

%%%%%%%%%%%%%%%%%%%%%%%%%%%%%%%%%%%%%%%%%%%%%%%%%%%%%%%%%%%%
% PROBLEM 17 — Lewis Dot (TikZ)
%%%%%%%%%%%%%%%%%%%%%%%%%%%%%%%%%%%%%%%%%%%%%%%%%%%%%%%%%%%%
\begin{problem}
Which of the following Lewis dot diagrams is correct?
\end{problem}

\begin{answer}
Correct: C \\

Below is a Lewis structure for CO showing the correct triple bond and lone pairs:

\begin{center}
\chemfig{\charge{45=\|,225=\|}{C}~\charge{45=\|,225=\|}{O}}
\end{center}

Note: Carbon monoxide has a triple bond between C and O, with a lone pair on each atom.

\end{answer}



%%%%%%%%%%%%%%%%%%%%%%%%%%%%%%%%%%%%%%%%%%%%%%%%%%%%%%%%%%%%
% PROBLEM 18
%%%%%%%%%%%%%%%%%%%%%%%%%%%%%%%%%%%%%%%%%%%%%%%%%%%%%%%%%%%%
\begin{problem}
The molecular shape of BF3 is: \\
A. bent \\ B. pyramidal \\ C. tetrahedral \\ D. trigonal planar
\end{problem}

\begin{answer}
Correct: D \\
Boron has three bonds and zero lone pairs, giving trigonal planar shape.
\end{answer}



%%%%%%%%%%%%%%%%%%%%%%%%%%%%%%%%%%%%%%%%%%%%%%%%%%%%%%%%%%%%
% PROBLEM 19
%%%%%%%%%%%%%%%%%%%%%%%%%%%%%%%%%%%%%%%%%%%%%%%%%%%%%%%%%%%%
\begin{problem}
The molecular shape of silicon dioxide (SiO2) is: \\
A. linear \\ B. pyramidal \\ C. bent \\ D. trigonal planar
\end{problem}

\begin{answer}
Correct: A \\

A representation of linear O=Si=O:

\begin{center}
\chemfig{O=Si=O}
\end{center}

\end{answer}



%%%%%%%%%%%%%%%%%%%%%%%%%%%%%%%%%%%%%%%%%%%%%%%%%%%%%%%%%%%%
% PROBLEM 20
%%%%%%%%%%%%%%%%%%%%%%%%%%%%%%%%%%%%%%%%%%%%%%%%%%%%%%%%%%%%
\begin{problem}
Given the Lewis structure O=O, what is the total number of electrons shared 
between the two oxygen atoms? \\
A. 1 \\
B. 2 \\
C. 3 \\
D. 4
\end{problem}

\begin{answer}
Correct: D \\
A double bond contains 4 shared electrons.
\end{answer}



\newpage

%%%%%%%%%%%%%%%%%%%%%%%%%%%%%%%%%%%%%%%%%%%%%%%%%%%%%%%%%%%%
% PROBLEM 21 — OCTET VIOLATION (TikZ)
%%%%%%%%%%%%%%%%%%%%%%%%%%%%%%%%%%%%%%%%%%%%%%%%%%%%%%%%%%%%
\begin{problem}
In the Lewis structure of SOCl2, which atom violates the octet rule?
\end{problem}

\begin{answer}
Correct: A (sulfur) \\

A sketch of SOCl2 structure:

\begin{center}
\chemfig{S(=[:90]\charge{45=\|,225=\|,135=\|,315=\|}{O})(-[:210]\charge{45=\|,225=\|,135=\|,315=\|}{Cl})(-[:330]\charge{45=\|,225=\|,135=\|,315=\|}{Cl})}
\end{center}

Sulfur has 10 electrons around it (expanded octet).
\end{answer}



%%%%%%%%%%%%%%%%%%%%%%%%%%%%%%%%%%%%%%%%%%%%%%%%%%%%%%%%%%%%
% PROBLEM 22 — STRUCTURE OF C2H3Cl
%%%%%%%%%%%%%%%%%%%%%%%%%%%%%%%%%%%%%%%%%%%%%%%%%%%%%%%%%%%%
\begin{problem}
Which Lewis structure best represents C2H3Cl?
\end{problem}

\begin{answer}
Correct: B \\

Structure shows C=C double bond with three hydrogens and one chlorine:

\begin{center}
\chemfig{C(-[:90]H)(-[:270]H)=C(-[:270]H)(-[:90]Cl)}
\end{center}

\end{answer}



%%%%%%%%%%%%%%%%%%%%%%%%%%%%%%%%%%%%%%%%%%%%%%%%%%%%%%%%%%%%
% PROBLEM 23 — POLARITY
%%%%%%%%%%%%%%%%%%%%%%%%%%%%%%%%%%%%%%%%%%%%%%%%%%%%%%%%%%%%
\begin{problem}
Hexane (C6H14) and water do not form a solution. Which statement explains this phenomenon? \\
A. Hexane is polar and water is nonpolar. \\
B. Hexane is ionic and water is polar. \\
C. Hexane is nonpolar and water is polar. \\
D. Hexane is nonpolar and water is ionic.
\end{problem}

\begin{answer}
Correct: C \\
Nonpolar substances do not dissolve in polar solvents ("like dissolves like").
\end{answer}






%============ 24 ============%
\begin{problem}
Which of the Lewis structures below best represents the molecule CHF$_3$?
\end{problem}

\begin{answer}
Correct choice: A.

A correct Lewis structure for CHF$_3$ has carbon in the center with
four single bonds (three to F, one to H). Each F has three lone pairs
and H has none. A Lewis structure:

\begin{center}
\chemfig{C(-[:90]\charge{45=\|,225=\|,135=\|,315=\|}{F})(-[:210]\charge{45=\|,225=\|,135=\|,315=\|}{F})(-[:330]\charge{45=\|,225=\|,135=\|,315=\|}{F})(-[:270]H)}
\end{center}
\end{answer}


\newpage

%============ 25 ============%
\begin{problem}
Electronegativity is defined as the tendency of an atom to\\
A. donate electrons to other atoms in a chemical bond\\
B. share electrons equally with other atoms\\
C. lose its valence electrons to become an ion\\
D. attract electrons toward itself in a chemical bond
\end{problem}

\begin{answer}
Correct choice: D.

Electronegativity measures how strongly an atom attracts the shared
electrons in a bond.
\end{answer}


%============ 26 ============%
\begin{problem}
Based on its location on the periodic table, which of the following
elements should have the largest value for electronegativity?\\
A. lithium \qquad
B. oxygen \qquad
C. potassium \qquad
D. selenium
\end{problem}

\begin{answer}
Correct choice: B (oxygen).

Electronegativity increases across a period and decreases down a group;
among these options oxygen is closest to the top right of the table.
\end{answer}


%============ 27 ============%
\begin{problem}
Which formula represents a nonpolar molecule containing polar
covalent bonds?\\
A. H$_2$O \qquad
B. CCl$_4$ \qquad
C. NH$_3$ \qquad
D. H$_2$
\end{problem}

\begin{answer}
Correct choice: B (CCl$_4$).

Each C--Cl bond is polar, but the tetrahedral geometry is perfectly
symmetric, so the molecular dipoles cancel and the molecule is overall
nonpolar.
\end{answer}





%============ SHORT ANSWER Q1 ============%
\begin{problem}
Decide if each description represents \textbf{IONIC}
bonding or \textbf{COVALENT} bonding.
\end{problem}

\begin{answer}
\begin{enumerate}
  \item ``It is a non conductor of electricity, whether it exists as a
        solid, melted, or dissolved in water.'' \\
        \textbf{Covalent} (molecular compounds do not conduct in any state).

  \item ``It is a nonelectrolyte in the solid form, but it can become a
        good conductor when melted or dissolved in water.'' \\
        \textbf{Ionic} (ions are free to move in the molten or aqueous state).

  \item ``The building blocks of this type of compound are called
        molecules.'' \\
        \textbf{Covalent} (molecules are made of covalently bonded atoms).

  \item ``The electrons are transferred from one element to another to
        form this type of bond.'' \\
        \textbf{Ionic}.

  \item ``The electrons are shared in between elements in this type of
        bond.'' \\
        \textbf{Covalent}.
\end{enumerate}
\end{answer}
  






\end{document}
