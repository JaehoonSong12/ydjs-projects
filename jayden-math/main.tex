\documentclass{article}
\usepackage{xcolor}
\usepackage{enumitem}
\usepackage{amsthm}
\usepackage{amsmath}
\usepackage{amssymb}
\usepackage{environ}
\usepackage{tikz}
\usepackage{pgfplots}

\usepackage{mathtools}
\usepackage[
    top=    0.50in,
    bottom= 0.50in,
    left=   0.50in,
    right=  0.50in,
]{geometry}

\usepackage{hyperref}

% no indent
\setlength\parindent{0pt}

% Theorem environments
\theoremstyle{definition}
\newtheorem{definition}{Definition}
\newtheorem{theorem}{Theorem}
\newtheorem{lemma}{Lemma}
\newtheorem{proposition}{Proposition}
\newtheorem{example}{Example}
\newtheorem{problem}{Problem}
\newtheorem{corollary}{Corollary}

\NewEnviron{answer}{%
  \par\noindent\vspace{10pt}%
  \textcolor{red}{\textbf{Answer}:}
  \par\noindent\textcolor{red}{\BODY}%
  \par\vspace{-0.5\baselineskip}\raggedleft\textcolor{red}{$\blacksquare$}%
  \par\vspace{10pt}
}

\author{Jayden}
\title{Emergency Note \\ Lecture Notes}
\date{\today}

\begin{document}

\maketitle

\section*{Polynomials: Even vs. Odd}

\begin{definition}
A function \( f(x) \) is said to be \textbf{even} if for all values of \( x \) in the domain, we have:
\[
f(-x) = f(x)
\]
\end{definition}

\begin{definition}
A function \( f(x) \) is said to be \textbf{odd} if for all values of \( x \) in the domain, we have:
\[
f(-x) = -f(x)
\]
\end{definition}

\begin{example}
Consider the function \( f(x) = x^2 + 4 \).

To check if it is even:
\[
f(-x) = (-x)^2 + 4 = x^2 + 4 = f(x)
\]
Since \( f(-x) = f(x) \), the function is \textbf{even}.
\end{example}

\begin{example}
Consider the function \( f(x) = x^3 - 3x \).

To check if it is odd:
\[
f(-x) = (-x)^3 - 3(-x) = -x^3 + 3x = -(x^3 - 3x) = -f(x)
\]
Since \( f(-x) = -f(x) \), the function is \textbf{odd}.
\end{example}

\begin{problem}
Determine if the following function is even, odd, or neither:
\[
f(x) = x^3 + x^2
\]
\end{problem}

\begin{answer}
To check if \( f(x) = x^3 + x^2 \) is even or odd:
\[
f(-x) = (-x)^3 + (-x)^2 = -x^3 + x^2
\]
Since \( f(-x) \neq f(x) \) and \( f(-x) \neq -f(x) \), the function is \textbf{neither even nor odd}.
\end{answer}





\newpage

\section*{Rational Functions}

\begin{definition}
A \textbf{rational function} is a function of the form
\[
f(x) = \frac{g(x)}{k(x)},
\]
where \( g(x) \) and \( k(x) \) are polynomial functions and \( k(x) \neq 0 \).
\end{definition}

\section*{1. Domain Restrictions}

A rational function is \textbf{undefined} whenever the denominator equals zero.

To determine the domain:
\begin{enumerate}
    \item Set the denominator equal to zero.
    \item Solve for the values of \( x \).
    \item Exclude these values from the domain.
\end{enumerate}

\begin{example}
Determine the domain of
\[
f(x) = \frac{3x + 2}{x^2 - 5x - 6}.
\]

Factor the denominator:
\[
x^2 - 5x - 6 = (x - 6)(x + 1).
\]

Thus, the domain excludes:
\[
x \neq 6,\ -1.
\]
\end{example}

\begin{example}
Determine the domain of
\[
v(x) = \frac{x}{x^2 - 3x + 2}.
\]

Factor the denominator:
\[
x^2 - 3x + 2 = (x - 2)(x - 1).
\]

Thus, the domain excludes:
\[
x \neq 2,\ 1.
\]
\end{example}

\begin{problem}
Determine the domain of
\[
h(x) = \frac{x^2 - 9}{x^2 - 4x - 5}.
\]
\end{problem}

\begin{answer}
Factor the denominator:
\[
x^2 - 4x - 5 = (x - 5)(x + 1).
\]
Thus, the domain excludes the values that make the denominator zero:
\[
x \neq 5,\ -1.
\]
\end{answer}

\section*{2. Zeros of Rational Functions}

\begin{definition}
A rational function equals zero when the \textbf{numerator} is zero and the \textbf{denominator is nonzero}.
\end{definition}

\begin{example}
Find the zeros of
\[
f(x) = \frac{3x + 2}{x^2 - 5x + 6}.
\]

Solve the numerator:
\[
3x + 2 = 0 \quad \Rightarrow \quad x = -\frac{2}{3}.
\]

Since this value does not make the denominator zero, it is a valid zero.
\end{example}

\begin{example}
Find the zeros of
\[
g(x) = \frac{x^2 - 4}{x^2 - 3x + 2}.
\]

Factor the numerator:
\[
x^2 - 4 = (x - 2)(x + 2),
\]
so potential zeros are \( x = \pm 2 \).

However, the denominator
\[
x^2 - 3x + 2 = (x - 2)(x - 1)
\]
excludes \( x = 2 \) and \( x = 1 \).

Thus the only valid zero is:
\[
x = -2.
\]
\end{example}

\begin{problem}
Find the zeros of
\[
p(x) = \frac{2x^2 - 8x}{x^2 - x - 6}.
\]
\end{problem}

\begin{answer}
Factor both numerator and denominator:
\[
2x^2 - 8x = 2x(x - 4), \qquad x^2 - x - 6 = (x - 3)(x + 2).
\]
The numerator is zero when \( x = 0 \) or \( x = 4 \). Neither value makes the denominator zero, so both are valid zeros:
\[
x = 0,\ 4.
\]
\end{answer}

\section*{3. Key Ideas}

\begin{itemize}
    \item A rational function is undefined where the denominator equals zero.
    \item A rational function is zero where the numerator equals zero (and denominator is nonzero).
    \item End behavior depends on how the numerator and denominator behave for large values of \( x \).
\end{itemize}









\end{document}
